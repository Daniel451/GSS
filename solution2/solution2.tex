\documentclass[ngerman]{fbi-aufgabenblatt}


\usepackage[autostyle=true,german=quotes]{csquotes}

\renewcommand{\Aufgabenblatt}{1}
\renewcommand{\Gruppe}{G01}
\renewcommand{\KleinGruppe}{A}
\renewcommand{\Teilnehmer}{Huynh, Krempels, Rupnow, Speck}

\begin{document}




\aufgabe{Grundlagen von Betriebssystemen}

(a) \quad Das Betriebssystem als Betriebsvermittler:
\\
Das Betriebssystem �bernimmt die Allokation, also die Vermittlung/Buchung von Systemressourcen,
dies schlie�t beispielsweise Speicher (z.B. RAM) oder Rechenzeit auf dem Hauptprozessor ebenso ein
wie die Nutzung von Systemger�ten, Netzwerkschnittstellen, usw.
\\
\\
Betriebssystem als virtuelle Maschine:
\\
Um eine zunehmend "komfortable" Nutzung der Systemeigenschaften zu erm�glichen muss
von der Hardwareebene abstrahiert werden. Das Betriebssystem wird dabei als eine Art virtuelle
Maschine t�tig. Ohne diese Abstraktion m�sste Software explizit f�r die aktuell verwendete Hardware
kompiliert werden und der Benutzer zu jeder Zeit alle Hardwareschichten kennen und mitunter verwalten.
\\
\\
(b) \quad Das Betriebssystem als Betriebsvermittler hat folgende Aufgaben:

1. Verwaltung der Betriebsmittel, also den Bedarf von Prozessen auf Anfrage hin abzudecken, sowie die Betriebsmittel m�glichst effektiv zu verteilen.

2. Es k�nnen mehere Programme auf die selbe Systemressource zugreifen; daher muss das Betriebsystem mit Konflikten �ber die Betriebsmittlel umgehen k�nnen. Beispielsweise eine Deadlock Situation, wie in der Vorlesung beschrieben wurde.
\\
\\
Das Betriebssystem als virtuelle Maschine hat folgende Aufgaben:

1. Die Nutzung eines Rechners m�glichst einfach zu gestalten, es sollten also spezifische Informationen �ber die vorhandene Hardware "versteckt" werden, um dem Benutzer die Arbeit/Verwaltung mit dem System zu erleichtern. 

2. Ebenso z�hlt dazu die Vermittlung zwischen Hardware und Software, damit nicht jede Applikation spezifisch f�r die aktuelle Architektur neu kompiliert werden muss.



\end{document}

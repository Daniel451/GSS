\documentclass[ngerman]{fbi-aufgabenblatt}


\usepackage[autostyle=true,german=quotes]{csquotes}
\usepackage{listings}

\renewcommand{\Aufgabenblatt}{3}
\renewcommand{\Gruppe}{G01}
\renewcommand{\KleinGruppe}{A}
\renewcommand{\Teilnehmer}{Huynh, Krempels, Rupnow, Speck}

\begin{document}




\aufgabe{Rechnersicherheit}

(a) \quad 
Zugangskontrolle: Ein System �berpr�ft f�r einen bestimmten Nutzer, ob dieser f�r die Benutzung von Betriebsmitteln des Systems authentisiert ist, also z.B. durch die Eingabe von Benutzername und Passwort oder auch durch eine Chipkarte erkannt werden. Damit soll verhindert werden,dass unautorisierte Personen mit dem System interagieren k�nnen.

Zugriffskontrolle: Wenn nun eine Person Zugang auf ein System erlangt hat, hei�t es nicht automatisch, dass dieser auch alle Operationen darauf ausf�hren kann. Vor der Ausf�hrung wird immer �berpr�ft ob der Nutzer auch die Rechte besitzt eine bestimmte Operation auszuf�hren.Dabei gibt es verschiedene Modelle um den Zugriff zu kontrollieren, z.B. die Mandatory Access Control (MAC) wo jeder Nutzer einer bestimmten Schutzebenezugewiesen, diese hierarchisch angeordnet ist.

(b) \quad
Nein es ist nicht sinnvoll ein System mit Zugangskontrolle aber ohne Zugriffskontrolle zu gestalten, da nicht jeder Nutzer Zugriff auf alle Daten und Operationen haben soll.
Ein Beispiel w�re wenn ein Kunde bei einem Bankautomaten Geld abheben will. Er benutzt seine Chipkarte um sich zu identifizieren und gibt eine PIN ein um best�tigen, dass ihm diese Karte auch geh�rt. H�tte das System nun keine Zugriffskontrollen implementiert, k�nnte der Nutzer z.B. sich die Daten von anderen Kunden ansehen und bearbeiten oder sein Konto mit Geldauff�llen, obwohl dieser nichts eingezahlt hat

(c) \quad
Wenn sich bei der Zugangskontrolle kein Nutzer identifiziert, kann das System bei der Zugriffskontrolle auch nicht bestimmen,welche Rechte dieser Nutzer nun hat, da dieser sich anonym angemeldet hat.

(d) \quad  
Mit dem Share-this-Folder-Link wurden implizit die Rechte (Leserechte, aber keine Schreibrechte) eines Nutzers festgelegt. Damit wei� das System, wenn jemand auf den Link zugreift, dass dieser Nutzer eben nur diese Rechte haben soll.

\aufgabe{Timing-Attack}

(1) \quad Diese Methode ist anf�llig f�r Timing-Angriffe, da die Bearbeitungszeit mit der Passwortl�nge korreliert. Eine triviale M�glichkeit w�re einen zufallsgenerierten Integer
zu erzeugen f�r jeden Schleifendurchlauf (also jedes Zeichen im Passwort) und dann wait() mit dem eben erzeugten Zufallsinteger auszuf�hren. Dadurch wird bei jedem Schleifendurchlauf
eine zuf�llige Zeiteinheit lang gewartet, wodurch ein Timing-Angriff abgewehrt werden k�nnte.

(2) \quad Einen Passwort�berpr�fungsalgorithmus angenommen vergleicht der Angreifer die Bearbeitungszeit f�r die �berpr�fung mit einer selbsterstellten Heuristik. Er wird dann Passw�rter genau der L�nge zuerst ausprobieren, die als aufgrund der Bearbeitungszeit als wahrscheinlich gilt.


\end{document}

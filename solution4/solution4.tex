\documentclass[ngerman]{fbi-aufgabenblatt}


\usepackage[autostyle=true,german=quotes]{csquotes}
\usepackage{listings}
\usepackage{blockgraph}

\renewcommand{\Aufgabenblatt}{4}
\renewcommand{\Gruppe}{G01}
\renewcommand{\KleinGruppe}{A}
\renewcommand{\Teilnehmer}{Huynh, Krempels, Rupnow, Speck}

\begin{document}



\aufgabe{Scheduling-Algorithmen}

\section*{Aufgabe 1a}
Scheduling-Strategie: Beste Bedieng�te

\begin{blockgraph}{28}{1}{0.4}
	\bglabelxx{0}
	\bglabelxx{5}
	\bglabelxx{10}
	\bglabelxx{15}
	\bglabelxx{20}
	\bglabelxx{25}
	
	\bgemptysingleblock{0}
	\bgblock{1}{7}{$P_1$}
	\bgemptysingleblock{7}
	\bgblock{8}{10}{$P_4$}
	\bgemptysingleblock{10}
	\bgblock{11}{19}{$P_5$}
	\bgemptysingleblock{19}
	\bgblock{20}{25}{$P_2$}
	\bgemptysingleblock{25}
	\bgblock{26}{27}{$P_3$}
\end{blockgraph}

\section*{Aufgabe 1b}
Scheduling-Strategie: Round Robin $\Delta t = 2$

\begin{blockgraph}{32}{1}{0.4}
	\bglabelxx{0}
	\bglabelxx{5}
	\bglabelxx{10}
	\bglabelxx{15}
	\bglabelxx{20}
	\bglabelxx{25}
	\bglabelxx{30}
	
	\bgemptysingleblock{0}
	\bgblock{1}{3}{$P_1$}
	\bgblock{3}{5}{$P_1$}
	\bgemptysingleblock{5}
	\bgblock{6}{8}{$P_2$}
	\bgemptysingleblock{8}
	\bgblock{9}{11}{$P_1$}
	\bgemptysingleblock{11}
	\bgblock{12}{13}{$P_3$}
	\bgemptysingleblock{13}
	\bgblock{14}{16}{$P_4$}
	\bgemptysingleblock{16}
	\bgblock{17}{19}{$P_2$}
	\bgemptysingleblock{19}
	\bgblock{20}{22}{$P_5$}
	\bgemptysingleblock{22}
	\bgblock{23}{24}{$P_2$}
	\bgemptysingleblock{24}
	\bgblock{25}{27}{$P_5$}
	\bgblock{27}{29}{$P_5$}
	\bgblock{29}{31}{$P_5$}
\end{blockgraph}

\section*{Aufgabe 1c}
Scheduling-Strategie: Round Robin $\Delta t = 4$

\begin{blockgraph}{31}{1}{0.4}
	\bglabelxx{0}
	\bglabelxx{5}
	\bglabelxx{10}
	\bglabelxx{15}
	\bglabelxx{20}
	\bglabelxx{25}
	\bglabelxx{30}
	
	\bgemptysingleblock{0}
	\bgblock{1}{5}{$P_1$}
	\bgemptysingleblock{5}
	\bgblock{6}{10}{$P_2$}
	\bgemptysingleblock{10}
	\bgblock{11}{13}{$P_1$}
	\bgemptysingleblock{13}
	\bgblock{14}{15}{$P_3$}
	\bgemptysingleblock{15}
	\bgblock{16}{18}{$P_4$}
	\bgemptysingleblock{18}
	\bgblock{19}{23}{$P_5$}
	\bgemptysingleblock{23}
	\bgblock{24}{25}{$P_2$}
	\bgemptysingleblock{25}
	\bgblock{26}{30}{$P_5$}
\end{blockgraph}

\section*{Aufgabe 1d}
Antwortzeiten der Prozesse
\begin{center}
	\begin{tabular}{l c c c c c c}
		Aufgabe & $P_1$ & $P_2$ & $P_3$ & $P_4$ & $P_5$ & $\O{}$\\
		\hline
		1a & $6$ & $20$ & $21$ & $3$ & $10$ & $\frac{60}{5} = 12$\smallskip\\
		1b & $10$ & $19$ & $7$ & $9$ & $22$ & $\frac{67}{5} = 13,4$\smallskip\\
		1c & $12$ & $20$ & $9$ & $11$ & $21$ & $\frac{73}{5} = 14,6$
	\end{tabular}
\end{center}

\section*{Aufgabe 2a}
Wenn die Deadlines aller Auftr�ge auf unbeschr�nkte Zeit eingehalten werden k�nnten, so muss es ein Zeit-Intervall geben, das (1) genau so lang ist, dass es ganzzahlig durch die Periodendauer aller Auftr�ge teilbar ist und (2) genug Platz f�r die Summe der  ben�tigten Bedienzeiten der Auftr�ge im Intervall bietet.
\begin{center}
	L�nge des Intervalls: $kgV(4;7;3) = 4 * 7 * 3 = 84$\bigskip\\
	\begin{tabular}{l c c c}
		Aufgabe & Anz. Perioden & & Ben�tigte Bedienzeit\\\hline
		$A_1:$ & $84 : 4 = 21$ & $\rightarrow$ & $28 * 1 = 28$\\
		$A_2:$ & $84 : 7 = 12$ & $\rightarrow$ & $21 * 1 = 21$\\
		$A_3:$ & $84 : 3 = 28$ & $\rightarrow$ & $12 * 3 = 36$
	\end{tabular}\bigskip\\
	Gesamte ben�tigte Bedienzeit: $28 + 21 + 36 = 85$
\end{center}
Da insgesamt mehr Bedienzeit ben�tigt wird als im Intervall verf�gbar ist, k�nnen die Deadlines selbst von einem idealen Scheduler nicht eingehalten werden.

\section*{Aufgabe 2b i}

\begin{blockgraph}{25}{5}{0.4}
	\bglabelxx{0}
	\bglabelxx{5}
	\bglabelxx{10}
	\bglabelxx{15}
	\bglabelxx{20}
	
	\bglabely{4}{$B_4$}
	\bgblock[4]{0}{4}{}
	\bgblock[4]{4}{8}{}
	\bgblock[4]{8}{12}{}
	\bgblock[4]{12}{16}{}
	\bgblock[4]{16}{20}{}
	\bgblock[4]{20}{24}{}
	\bgblock[4]{24}{25}{}
	
	\bglabely{3}{$B_1$}
	\bgblock[3]{0}{7}{}
	\bgblock[3]{7}{14}{}
	\bgblock[3]{14}{21}{}
	\bgblock[3]{21}{25}{}
	
	\bglabely{2}{$B_3$}
	\bgblock[2]{0}{9}{}
	\bgblock[2]{9}{18}{}
	\bgblock[2]{18}{25}{}
	
	\bglabely{1}{$B_2$}
	\bgblock[1]{0}{11}{}
	\bgblock[1]{11}{22}{}
	\bgblock[1]{22}{25}{}
	
	\bglabely{0}{CPU}
	\bgblock[0]{0}{1}{$B_4$}
	\bgblock[0]{1}{4}{$B_1$}
	\bgblock[0]{4}{5}{$B_4$}
	\bgblock[0]{5}{7}{$B_3$}
	\bgblock[0]{7}{8}{$B_2$}
	\bgblock[0]{8}{9}{$B_4$}
	\bgblock[0]{9}{12}{$B_1$}
	\bgblock[0]{12}{13}{$B_4$}
	\bgblock[0]{13}{15}{$B_3$}
	\bgblock[0]{15}{16}{$B_2$}
	\bgblock[0]{16}{17}{$B_4$}
	\bgblock[0]{17}{20}{$B_1$}
	\bgblock[0]{20}{21}{$B_4$}
	\bgblock[0]{21}{24}{$B_1$}
	\bgblock[0]{24}{25}{$B_4$}
\end{blockgraph}

Die ersten vier Zeilen visualisieren die Periodendauer der einzelnen Auftr�ge.

\section*{Aufgabe 2c}

\begin{blockgraph}{15}{4}{0.4}
	\bglabelxx{0}
	\bglabelxx{5}
	\bglabelxx{10}
	
	\bglabely{0}{CPU 0}
	\bgblock[0]{0}{4}{$P_1$}
	\bgblock[0]{4}{5}{$P_2$}
	\bgblock[0]{5}{10}{$P_6$}
	\bgblock[0]{10}{14}{$P_7$}
	
	\bglabely{1}{CPU 1}
	\bgblock[1]{2}{7}{$P_3$}
	\bgblock[1]{7}{11}{$P_2$}
	
	\bglabely{2}{CPU 2}
	\bgblock[2]{2}{3}{$P_2$}
	\bgblock[2]{3}{11}{$P_5$}
	
	\bglabely{3}{CPU 3}
	\bgblock[3]{3}{11}{$P_4$}
\end{blockgraph}\\
Da die Aufgabenstellung etwas schwammig formuliert war, haben wir angenommen, dass eine kleinere Zahl in der Priorit�t-Zeile eine h�here Priorit�t bedeutet, so hat zum Beispiel $P_3$ eine h�here Priorit�t als $P_2$.

\section*{Aufgabe 3a}
Aus Gr�nden der Darstellung sind s�mtliche Werte durch 5 dividiert.
Dies f�hrt zu folgender Tabelle:
\begin{center}
	\begin{tabular}{l c c c}
		Auftrag & $M$ & $B$ & $Z$\\\hline
		Periodendauer $p_i$ & $23$ & $8$ & $12$\\
		Bedienzeit $b_i$ & $6$ & $2$ & $4$
	\end{tabular}
\end{center}
und der Darstellung im Intervall $[0,34]$ mit $Prio(B) > Prio(Z) > Prio(M)$.\\
\begin{blockgraph}{34}{5}{0.4}
	\bglabelxx{0}
	\bglabelxx{5}
	\bglabelxx{10}
	\bglabelxx{15}
	\bglabelxx{20}
	\bglabelxx{25}
	\bglabelxx{30}
	\bglabelxx{34}
	
	\bglabely{4}{M}
	\bgblock[4]{0}{6}{}
	% \bgemptyblock[4]{6}{23}
	\bgblock[4]{23}{29}{}
	% \bgemptyblock[4]{29}{34}
	
	\bglabely{3}{Z}
	\bgblock[3]{0}{4}{}
	% \bgemptyblock[3]{0}{4}
	\bgblock[3]{12}{16}{}
	% \bgemptyblock[3]{12}{16}
	\bgblock[3]{24}{28}{}
	% \bgemptyblock[3]{24}{28}
	
	\bglabely{2}{B}
	\bgblock[2]{0}{2}{}
	% \bgemptyblock[2]{0}{2}
	\bgblock[2]{8}{10}{}
	% \bgemptyblock[2]{8}{10}
	\bgblock[2]{16}{18}{}
	% \bgemptyblock[2]{16}{18}
	\bgblock[2]{24}{26}{}
	% \bgemptyblock[2]{24}{26}
	\bgblock[2]{32}{34}{}
	% \bgemptyblock[2]{32}{34}
	
	\bglabely{1}{Lock}
	\bgblock[1]{0}{2}{$B$}
	\bgemptyblock[1]{2}{6}
	\bgblock[1]{6}{12}{$M$}
	\bgblock[1]{12}{14}{$B$}
	\bgemptyblock[1]{14}{16}
	\bgblock[1]{16}{18}{$B$}
	\bgemptyblock[1]{18}{23}
	\bgblock[1]{23}{33}{$M$}
	\bgblock[1]{33}{34}{$B$}
	
	\bglabely{0}{CPU}
	\bgblock[0]{0}{2}{$B$}
	\bgblock[0]{2}{6}{$Z$}
	\bgblock[0]{6}{12}{$M$}
	\bgblock[0]{12}{14}{$B$}
	\bgblock[0]{14}{16}{$Z$}
	\bgblock[0]{16}{18}{$B$}
	\bgblock[0]{18}{20}{$Z$}
	\bgemptyblock[0]{20}{23}
	\bgblock[0]{23}{24}{$M$}
	\bgblock[0]{24}{28}{$Z$}
	\bgblock[0]{28}{33}{$M$}
	\bgblock[0]{33}{34}{$B$}
\end{blockgraph}\\
Zeitpunkt 12: B wird verz�gert ausgef�hrt, da in $[6,12]$ M die Semaphore verwendet.\\
Zeitpunkt 16: Z wird von B verdr�ngt.\\
Zeitpunkt 24: M wird von Z verdr�ngt, Semaphore weiterhin gesperrt.\\
Die letzte Verdr�ngung von M durch Z f�hrt dazu, dass in $[24,32]$ Auftrag B seine Deadline nicht einhalten kann.

\section*{Aufgabe 3b}
Als Priorit�tsinversion wird die unbeabsichtigte Umkehrung der Priorit�ten bezeichnet: Auftrag B konnte Z nicht verdr�ngen, da die von B ben�tigte Semaphore noch von M verwendet wurde, sodass Z f�r eine kurze Zeit eine h�here Priorit�t als B hatte.

\end{document}

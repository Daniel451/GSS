\documentclass[ngerman]{fbi-aufgabenblatt}


\usepackage[autostyle=true,german=quotes]{csquotes}
\usepackage{listings}
\usepackage{blockgraph}
\usepackage{amsmath}

\renewcommand{\Aufgabenblatt}{4}
\renewcommand{\Gruppe}{G01}
\renewcommand{\KleinGruppe}{A}
\renewcommand{\Teilnehmer}{Huynh, Krempels, Rupnow, Speck}

\begin{document}



\aufgabe{Scheduling-Algorithmen}

(1a) \quad Scheduling-Strategie: Beste Bedieng�te

\begin{blockgraph}{15}{1}{1}
	\bglabelxx{0}
	\bglabelxx{5}
	\bglabelxx{10}
	\bglabelxx{15}

	\bgblock{0}{6}{$P_1$}
	\bgblock{7}{8}{$P_3$}
	\bgblock{9}{11}{$P_4$}
	\bgblock{12}{15}{$P_2$}
\end{blockgraph}
\begin{blockgraph}{15}{1}{1}
	\bglabelx{0}{15}
	\bglabelx{5}{20}
	\bglabelx{10}{25}
	\bglabelx{15}{30}
	
	\bgblock{0}{2}{$P_2$}
	\bgblock{3}{11}{$P_5$}
\end{blockgraph}
\\
Als Bedienzeit in der Formel f�r die Bedieng�te haben wir die Bedienzeitforderung angenommen. Abgesehen davon fehlt eine Definition, ob die Bedieng�te proportional oder anti-proportional ist. Wir gehen davon aus, dass eine hohe Bedieng�te eine hohe Priorit�t impliziert.
\\
\\
(1b) \quad Round Robin: $\Delta t = 2$
\\
\begin{blockgraph}{15}{1}{1}
	
	
	
	\bglabelxx{0}
	\bglabelxx{5}
	\bglabelxx{10}
	\bglabelxx{15}
	
	\bgblock{0}{2}{$P_1$}
	\bgblock{2}{4}{$P_1$}
	
	\bgblock{4}{5}{$P_1$}
	\bgemptysingleblock{5}
	
	\bgblock{6}{7}{$P_2$}
	\bgemptysingleblock{7}
	
	\bgblock{8}{9}{$P_1$}
	\bgemptysingleblock{9}
	
	\bgblock{10}{11}{$P_3$}
	\bgemptysingleblock{11}
	
	\bgblock{12}{13}{$P_2$}
	\bgemptysingleblock{13}
	
	\bgblock{14}{15}{$P_4$}
	
	
	% 6 : P2 - P1 - P3
	% 7 : (P1 - P3 - P2 - P4)
	% 8 : P1 - P3 - P2 - P4
	% 9 : (P3 - P2 - P4 - P5)
	% 10: P3 - P2 - P4 - P5
	% 11: (P2 - P4 - P5)
	% 12: P2 - P4 - P5
	% 13: (P4 - P5 - P2)
	% 14: P4 - P5 - P2
	% 15: (P5 - P2 - P4)
	% 16: P5 - P2 - P4
	% 17: (P2 - P4 - P5)
	% 18: P2 - P4 - P5
	% 19: (P4 - P5 - P2)
	% 20: P4 - P5 - P2
	% 21: (P5 - P2)
	% 22: P5 - P2
	% 23: (P2 - P5)
	% 24: P2 - P5
	% 25: (P5 - P2)
	% 26: P5 - P2
	% 27: (P2 - P5)
	% 28: P2 - P5
	% 29: (P5)
	% 30: P5
	
\end{blockgraph}
\begin{blockgraph}{15}{1}{1}
	\bglabelx{0}{15}
	\bglabelx{5}{20}
	\bglabelx{10}{25}
	\bglabelx{15}{30}
	
	\bgemptysingleblock{0}
	
	\bgblock{1}{2}{$P_5$}
	\bgemptysingleblock{2}
	
	\bgblock{3}{4}{$P_2$}
	\bgemptysingleblock{4}
	
	\bgblock{5}{6}{$P_4$}
	\bgemptysingleblock{6}
	
	\bgblock{7}{8}{$P_5$}
	\bgemptysingleblock{8}
	
	\bgblock{9}{10}{$P_2$}
	\bgemptysingleblock{10}
	
	\bgblock{11}{12}{$P_5$}
	\bgemptysingleblock{12}
	
	\bgblock{13}{14}{$P_2$}
	\bgemptysingleblock{14}
\end{blockgraph}
\begin{blockgraph}{15}{1}{1}
	\bglabelx{0}{30}
	\bglabelx{5}{35}
	\bglabelx{10}{40}
	\bglabelx{15}{45}
	
	\bgblock{0}{2}{$P_5$}
	\bgblock{2}{4}{$P_5$}
	\bgblock{4}{5}{$P_5$}
\end{blockgraph}





\aufgabe{Echtzeit \& Mehrprozessor-Scheduling}

(2a) \quad Um Deadlines multipler Auftr�ge jederzeit einhalten zu k�nnen, m�sste der besagte, ideale Scheduler ein Zeitintervall finden, welches stets ausreichend Kapazit�ten f�r die Summe der Bedienzeitforderungen aller Auftr�ge bietet.
\\
Ferner w�re es n�tig, dass dieses Zeitintervall auch noch ganzzahlig, also ohne Divisionsrest, teilber durch die kumulierte Periodendauer aller Auftr�ge ist.
\\
\\
Die L�nge des Intervalls w�rde dabei dem kleinsten, gemeinsamen Vielfachen der Periodendauer aller Auftr�ge entsprechen:
\\
\begin{equation*}
	kgV = 4 * 7 * 3 = 84  
\end{equation*}
\\
Danach ergibt sich f�r die ben�tigte Bedienzeit f�r die einzelnen Auftr�ge durch:
\\
\begin{equation*}
	\frac{\text{L�nge-Intervall}}{\text{Periodendauer vom Auftrag}} = \text{Anzahl Perioden}
\end{equation*}
\\
Und die insgesamt ben�tigte Bedienzeit durch:
\\
\begin{equation*}
	\text{Anzahl Perioden} * \text{Bedienzeitforderung pro Periode} = \text{Gesamtbedienzeit}
\end{equation*}
\\
Daraus ergibt sich f�r die Auftr�ge $A_1$, $A_2$ und $A_3$ folgendes:
\begin{center}
	\begin{tabular}{l c c}
		Aufgabe & Anz. Perioden & Ben�tigte Bedienzeit\\\hline
		\\
		$A_1$ & $\frac{84}{4} = 21$  & $21 * 1 = 21$
		\\
		\\
		$A_2$ & $\frac{84}{7} = 12$  & $12 * 3 = 36$
		\\
		\\
		$A_3$ & $\frac{84}{3} = 28$  & $28 * 1 = 28$
	\end{tabular}\bigskip\\
	Daraus resultiert die Bedienzeit insgesamt: $21 + 36 + 28 = 85$
\end{center}
Insgesamt w�rde damit also mehr Bedienzeit ben�tigt, als das Intervall Kapazit�ten hat verf�gbar hat.
\\
Die Deadlines w�rden also auch von einem idealen Scheduler nicht eingehalten werden k�nnen.
\\
\\
(2c)

\begin{blockgraph}{15}{4}{1}
	\bglabelxx{0}
	\bglabelxx{5}
	\bglabelxx{10}
	\bglabelxx{15}
	
	\bglabely{0}{CPU 0}
	\bgblock[0]{0}{4}{$P_1$}
	\bgblock[0]{4}{5}{$P_2$}
	\bgblock[0]{5}{10}{$P_6$}
	\bgblock[0]{10}{14}{$P_7$}
	
	\bglabely{1}{CPU 1}
	\bgblock[1]{2}{7}{$P_3$}
	\bgblock[1]{7}{11}{$P_2$}
	
	\bglabely{2}{CPU 2}
	\bgblock[2]{2}{3}{$P_2$}
	\bgblock[2]{3}{11}{$P_5$}
	
	\bglabely{3}{CPU 3}
	\bgblock[3]{3}{11}{$P_4$}
\end{blockgraph}


\aufgabe{Priorit�tsinversion}


(3a) \quad Die in der Aufgabe genannten Werte wurden mit 5 dividiert, um sie leichter darzustellen:
\begin{center}
	\begin{tabular}{l c c c}
		Auftrag & $M$ & $B$ & $Z$\\\hline
		Periodendauer $p_i$ & $23$ & $8$ & $12$\\
		Bedienzeit $b_i$ & $6$ & $2$ & $4$
	\end{tabular}
\end{center}
Es folgt die Intervallsdarstellung $[0,34]$ mit $Prio(B) > Prio(Z) > Prio(M)$:
\\
\\
\begin{blockgraph}{35}{5}{0.4}
	\bglabelxx{0}
	\bglabelxx{5}
	\bglabelxx{10}
	\bglabelxx{15}
	\bglabelxx{20}
	\bglabelxx{25}
	\bglabelxx{30}
	\bglabelxx{35}
	
	
	\bglabely{4}{M}
	\bgblock[4]{0}{6}{}
	% \bgemptyblock[4]{6}{23}
	\bgblock[4]{23}{29}{}
	% \bgemptyblock[4]{29}{34}
	
	\bglabely{3}{Z}
	\bgblock[3]{0}{4}{}
	% \bgemptyblock[3]{0}{4}
	\bgblock[3]{12}{16}{}
	% \bgemptyblock[3]{12}{16}
	\bgblock[3]{24}{28}{}
	% \bgemptyblock[3]{24}{28}
	
	\bglabely{2}{B}
	\bgblock[2]{0}{2}{}
	% \bgemptyblock[2]{0}{2}
	\bgblock[2]{8}{10}{}
	% \bgemptyblock[2]{8}{10}
	\bgblock[2]{16}{18}{}
	% \bgemptyblock[2]{16}{18}
	\bgblock[2]{24}{26}{}
	% \bgemptyblock[2]{24}{26}
	\bgblock[2]{32}{34}{}
	% \bgemptyblock[2]{32}{34}
	
	\bglabely{1}{Lock}
	\bgblock[1]{0}{2}{$B$}
	\bgemptyblock[1]{2}{6}
	\bgblock[1]{6}{12}{$M$}
	\bgblock[1]{12}{14}{$B$}
	\bgemptyblock[1]{14}{16}
	\bgblock[1]{16}{18}{$B$}
	\bgemptyblock[1]{18}{23}
	\bgblock[1]{23}{33}{$M$}
	\bgblock[1]{33}{34}{$B$}
	
	\bglabely{0}{CPU}
	\bgblock[0]{0}{2}{$B$}
	\bgblock[0]{2}{6}{$Z$}
	\bgblock[0]{6}{12}{$M$}
	\bgblock[0]{12}{14}{$B$}
	\bgblock[0]{14}{16}{$Z$}
	\bgblock[0]{16}{18}{$B$}
	\bgblock[0]{18}{20}{$Z$}
	\bgemptyblock[0]{20}{23}
	\bgblock[0]{23}{24}{$M$}
	\bgblock[0]{24}{28}{$Z$}
	\bgblock[0]{28}{33}{$M$}
	\bgblock[0]{33}{34}{$B$}
\end{blockgraph}
\\
\small{* bisher haben wir die Blockgraphen stets einem Skalierungsfaktor von 1 gezeichnet, da dies bei uns der einzig verwendete Modus war, in dem die Grafik nicht verzerrt aussieht (Bl�cke stets kleiner als 1x1 K�stchen, Zusammenh�nge schwerer zu erkennen). Bei dieser Aufgabe war uns dies allerdings zu umst�ndlich, da wir ansonsten 3 Blockgraphen h�tten verwenden m�ssen.}
\normalsize
\\
\\
In der letzten Verd�ngung von M durch Z wird als Konsequenz davon in $[24,32]$ die Deadline von Auftrag B nicht eingehalten.
\\
\\
Zeitpunkt 12 \quad Verz�gerte Ausf�hrung von B, weil in $[6,12]$ von M die Semaphore genutzt wird
\\
\\
Zeitpunkt 16 \quad B verd�ngt Z.
\\
\\
Zeitpunkt 24 \quad Z verd�ngt M, Semaphore ist gesperrt.


\end{document}

\documentclass[ngerman,parskip=half]{scrartcl}

\usepackage[latin1]{inputenc}
\usepackage{blockgraph}

\title{Verwendung von \texttt{blockgraph}}
\date{}

\begin{document}

\maketitle

Dieses Dokument zeigt die Verwendung des Pakets \texttt{blockgraph} am Beispiel von Prozessor-Schedules. Im Quellcode befinden sich Kommentare, die die Makros im Detail beschreiben.

\section*{Beispiel 1: Einprozessorsystem (ohne Prozesswechselzeit)}

% blockgraph-Umgebung. Erzeugt den Blockgraphen. Parameter:
%   Breite (X-Achse)
%   H�he (Y-Achse)
%   Skalierung (Vergr��erungsfaktor)
\begin{blockgraph}{15}{1}{0.4}
	% \bglabelxx erzeut Beschriftung der X-Achse an bestimmter Position
	\bglabelxx{0}
	\bglabelxx{5}
	\bglabelxx{10}
	\bglabelxx{15}

	% \bgblock erzeugt Block innerhalb des Graphen. Parameter:
	%    Y-Position (z.B. CPU), optional
	%    Beginn auf der X-Achse
	%    Ende auf der X-Achse
	%    Beschriftung
	\bgblock{0}{4}{$P_1$}
	\bgblock{4}{7}{$P_2$}
	\bgblock{7}{12}{$P_3$}

\end{blockgraph}



\section*{Beispiel 2: Multiprozessorsystem (mit Prozesswechselzeit)}

% blockgraph-Umgebung. Erzeugt den Blockgraphen. Parameter:
%   Breite (X-Achse)
%   H�he (Y-Achse)
%   Skalierung (Vergr��erungsfaktor)
\begin{blockgraph}{20}{4}{0.4}
	% \bglabelxx erzeut eine Beschriftung der X-Achse an bestimmter Position, wobei Position und Beschriftung identisch sind
	\bglabelxx{0}
	\bglabelxx{5}
	\bglabelxx{10}
	\bglabelxx{15}
	\bglabelxx{20}

	% \bglabely erzeut eine Beschriftung der X-Achse an bestimmter Position
	\bglabely{0}{CPU 0}
	\bglabely{1}{CPU 1}
	\bglabely{2}{CPU 2}
	\bglabely{3}{CPU 3}

	% \bgblock erzeugt einen Block innerhalb des Graphen. Parameter:
	%    Y-Position (z.B. CPU), optional
	%    Beginn auf der X-Achse
	%    Ende auf der X-Achse
	%    Beschriftung
	\bgblock[0]{1}{4}{$P_1$}
	\bgblock[0]{5}{7}{$P_4$}
	\bgblock[0]{8}{12}{$P_7$}

	% \bgemptysingleblock erzeugt leeren Block (z.B. Prozesswechsel). Parameter:
	%    Y-Position (z.B. CPU), optional
	%    Position auf der X-Achse
	\bgemptysingleblock[0]{0}
	\bgemptysingleblock[0]{4}
	\bgemptysingleblock[0]{7}

	\bgemptysingleblock[1]{0}
	\bgblock[1]{1}{8}{$P_2$}
	\bgemptysingleblock[1]{8}
	\bgblock[1]{9}{18}{$P_8$}

	\bgemptysingleblock[2]{1}
	\bgblock[2]{2}{6}{$P_3$}
	\bgemptysingleblock[2]{6}
	\bgblock[2]{7}{15}{$P_6$}

	\bgemptysingleblock[3]{4}
	\bgblock[3]{5}{15}{$P_5$}
\end{blockgraph}


\end{document}

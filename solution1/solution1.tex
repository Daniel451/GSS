\documentclass[ngerman]{fbi-aufgabenblatt}


\renewcommand{\Aufgabenblatt}{1}
\renewcommand{\Gruppe}{G01}
\renewcommand{\KleinGruppe}{A}
\renewcommand{\Teilnehmer}{Huynh, Krempels, Rupnow, Speck}

\begin{document}

\aufgabe{Allgemeine Aussagen zur IT-Sicherheit}

(1) \quad
Ein verteiltes System bietet insofern eine erh�hte Sicherheit, als dass das System aus mehreren Teilkomponenten besteht, eine Infizierung oder ein Sicherheitsbruch einer Teilkomponente also nichtnotwendigerweise alle Teilkomponenten betrifft, was bei einem nicht-verteilten System der Fall w�re.
\\
Ein Nachteil eines verteilten System gegen�ber eines nicht-verteilten ist der erh�hte Wartungs-/Installationsaufwand, da mehrere Systeme offensichtlich einen gr��eren Ressourcenverbrauch bedeuten als einzelne, in sich geschlossene Systeme.
\\
\\
(2) \quad
DAU vor dem Bildschirm. Nutzung unsicherer Software/Systeme aufgrund von Komfortabilit�t jener Software/Systeme (sichere Systeme sind selten einfach und komfortabel benutzbar). Hohe Sicherheitsstandards verursachen h�ufig hohe Kosten, es existiert also ein Tradeoff zwischen Sicherheit und Machbarkeit/Finanzierbarkeit.
\\
\\
(3) \quad





\end{document}

\documentclass[ngerman]{fbi-aufgabenblatt}


\renewcommand{\Aufgabenblatt}{1}
\renewcommand{\Gruppe}{G01}
\renewcommand{\KleinGruppe}{A}
\renewcommand{\Teilnehmer}{Huynh, Krempels, Rupnow, Speck}

\begin{document}

\aufgabe{Allgemeine Aussagen zur IT-Sicherheit}

(1) \quad
Ein verteiltes System bietet insofern eine erh�hte Sicherheit, als dass das System aus mehreren Teilkomponenten besteht, eine Infizierung oder ein Sicherheitsbruch einer Teilkomponente also nichtnotwendigerweise alle Teilkomponenten betrifft, was bei einem nicht-verteilten System der Fall w�re.
\\
Ein Nachteil eines verteilten System gegen�ber eines nicht-verteilten ist der erh�hte Wartungs-/Installationsaufwand, da mehrere Systeme offensichtlich einen gr��eren Ressourcenverbrauch bedeuten als einzelne, in sich geschlossene Systeme.
\\
\\
(2) \quad
DAU vor dem Bildschirm. Nutzung unsicherer Software/Systeme aufgrund von Komfortabilit�t jener Software/Systeme (sichere Systeme sind selten einfach und komfortabel benutzbar). Hohe Sicherheitsstandards verursachen h�ufig hohe Kosten, es existiert also ein Tradeoff zwischen Sicherheit und Machbarkeit/Finanzierbarkeit.
\\
\\
(3a) \quad
Da mehr Essen als normalerweise bestellt wird, ist der logische Schluss daraus, dass die Arbeitszeiten l�nger sind als sonst. Dies w�rde die Wahrscheinlichkeit von Fehlern durch �berarbeitung und/oder �berm�dung steigern.
\\
Die Angreifbarkeit gegen�ber Schadprogrammen oder Hacker aus dem Internet ist also erh�ht.
\\
Abgesehen davon k�nnte auch einer der zahlreichen Lieferanten Spionageabsichten haben. Beispielsweise k�nnte er versuchen mit einem WiFi-f�higen Ger�t Funkverkehr abzuh�ren oder mit Schadsoftware ausgestattete USB-Sticks in Umlauf zu bringen.
\\
In jedem Fall w�rde das Schutzziel der Vertraulichkeit also bedroht. Sollte ein Eindringen in das System per Schadsoftware gelingen, w�re ein ver�ndernder Zugriff denkbar, sodass auch die Integrit�t und Verf�gbarkeit als gef�hrdet einzustufen w�re.
\\
\\
(3b) \quad
Hier k�nnten alle drei Schutzziele beeinflusst werden. Bei einem �ffentlichen, nicht verschl�sselten WLAN-Netz w�re es durchaus m�glich, einen vorher manipulierten AP mit der gleichen SSID und entsprechender Schadsoftware auszustatten, auf die sich in der N�he befindliche Clients dann automatisch verbinden.
\\
Den �ber den AP laufenden Netzwerkverkehr k�nnte man dann loggen und sp�ter auswerten, damit w�re das Schutzziel der Vertraulichkeit verletzt.
\\
Ebenfalls w�re eine Echtzeitmanipulation der Verbindung denkbar, indem man ankommende Daten in einen Cache l�dt, diese ver�ndern und erst dann zu dem Zielhost �bermittelt.
\\
Auch das Schutzziel der Verf�gbarkeit ist gef�hrdet, da so einzelne Zielhost oder der komplette Datenverkehr blockiert werden k�nnten.




\end{document}
